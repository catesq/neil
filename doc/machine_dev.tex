\documentstyle{article}
\begin{document}
\title{Writing libneil Machines Tutorial}
\author{Vytautas Jancauskas}
\maketitle

\section{Introduction}
We are going to examine here how to write three types of libneil machines - an
effect, a controller and a generator. They are used respectively to
process audio, control parameters of other machines and generate
sounds. For an effect we will write a simple ring modulator with
selectable modulator waves, for controller - a logistic map based
controller and for generator - a simple lo-fi style synth. These
examples are meant to thoroughly demonstrate the capabilities of the
API and what kind of plug-ins you can write.

\section{Ring Modulator}

A ring modulator is basically a device that takes two audio sources
and multiplies them together. We will want one source to be built in
to the effect itself so this gives us a rough outline of the machine -
generate a waveform of some kind with a specified frequency, multiply
it with the input and voila.

Now let us decide what kind of a user interface we want. We will
probably want a frequency control for the modulator, and the effect
amount which will be the gain applied to the modulator. In effect we
will scale the modulator before multiplying it with the input. We
probably want our frequency slider to go from 20Hz to oh say 20000Hz
and we will want our gain slider to go from, for example, -96dB up to
0dB which will play the wave at full volume. Now that we have this
information we will want to make a basic set of C++ files to create a
stub for our machine. You should create a new directory, name it
``ringmod'' and create three files in it.
\end{document}
