\starttext
\setupbodyfont[sansserif, 10pt]
\setuppagenumbering [location=]

\title{SoMono Cloud}

\SetTableToWidth{\textwidth}

Cloud is a grain synthesis based synthesizer. It takes a wave file and 
plays back short parts of it with varying parameters that are chosen at 
random from a given range. Each such short part is called a grain. You
can select it's length, starting offset into the wave file, envelope 
parameters, playback rate, etc.

\placefigure{Envelope} {\externalfigure [grain_env.pdf][scale=500]}

In the figure the envelope used for the grains is shown. The envelope
would look that way if skew was set to 50\% and sustain was set to \%33.

\placetable{none}{
  \starttable[|l|l|l|l|p50|]
  \HL
  \NC \bf Parameter \NC \bf Min \NC \bf Max \NC \bf Units \NC \bf Meaning \NC \AR
  \HL
  \NC Wave \NC 1 \NC 199 \NC Index \NC This allows you to pick a
  loaded wave file from a wave slot by name. \NC \AR

  \NC Offset Mean \NC 0.0 \NC 100.0 \NC \% \NC The mean offset into
  the wave file. This is the percentage by which the playback is
  skipped in to the wave file. For example setting this to 50\% will
  start the playback from the middle of the file for each new grain. \NC \AR

  \NC Offset Devi \NC 0.0 \NC 1.0 \NC Real \NC This and other devi
  (deviation) parameters let you adjust how much will the actual value
  deviate randomly from the mean value that you set using the mean
  parameters. For example if offset mean value ranges from 0\% to 100\%
  then setting Offset Devi parameter to say 0.1 and setting Offset
  Mean to say 50\% then the actual value will be from the interval
  (40\%, 60\%) picked at random. This works the same way for all the
  other parameters. \NC \AR

  \NC Amp Mean \NC 0.0 \NC 1.0 \NC Real \NC Each grains output is
  multiplied by the value specified by this parameter thus scaling the
  volume down if it is set to something less than 1. You use it to
  either avoid clipping when alot of grains are playing at the same
  time or with the devi parameter to allow grains of different
  volumes. \NC \AR

  \NC Amp Devi \NC 0.0 \NC 1.0 \NC Real \NC Deviation value for the
  amp parameter. \NC \AR

  \NC Length Mean \NC 10.0 \NC 1000.0 \NC ms \NC The length for which
  each grain will be playing in miliseconds. \NC \AR

  \NC Length Devi \NC 0.0 \NC 1.0 \NC Real \NC Deviation value for the
  length parameter. \NC \AR

  \NC Sustain Mean \NC 0.0 \NC 100.0 \NC \% \NC This value stands for
  the percentage of time (which was specified with the length
  parameter) that wil be spent in the sustain stage. \NC \AR

  \NC Sustain Devi \NC 0.0 \NC 1.0 \NC Real \NC Deviation value for
  the sustain parameter. \NC \AR

  \NC Skew Mean \NC 0.0 \NC 100.0 \NC \% \NC Skew in this case means
  the amount of playback time that will be spent in the attack
  stage. If skew is set to 50\% than it means that both attack and
  decay stages will be given equal time. If it is set to 20\% it means
  that attack will be given 20\% of the time that is left after
  accounting for sustain and release will be given the remaining 80\%
  thus attack will be shorter than release. You get the picture. \NC \AR

  \NC Skew Devi \NC 0.0 \NC 1.0 \NC Real \NC Deviation value for the
  skew parameter. \NC \AR

  \NC Rate Mean \NC -4.0 \NC 4.0 \NC Real \NC Playback rate for each
  grain - 0.0 means that each grain plays at the original rate, 1.0
  means that the it is played twice as fast, -1.0 that it is played
  twice as slow etc. \NC \AR

  \NC Rate Devi \NC 0.0 \NC 1.0 \NC Real \NC Deviation value for the
  rate parameter. \NC \AR

  \NC Pan Mean \NC -1.0 \NC 1.0 \NC Real \NC The panorama position of
  the grain. Here -1.0 means panning hard left and 1.0 means panning
  hard right, the other values cover the rest of the panorama
  spectrum. \NC \AR

  \NC Pan Devi \NC 0.0 \NC 1.0 \NC Real \NC Deviation value for the
  pan parameter. \NC \AR

  \NC Density \NC 0.0 \NC 1.0 \NC Real \NC Probability that a grain
  will be triggered for playback at each given processing cycle. The
  bigger this value the more grains playing at the same time. \NC \AR

  \NC Grain \NC 1 \NC 64 \NC Integer \NC The maximum number of grains
  that can be playing that the same time. \NC \AR
  \HL
  \stoptable
}

\stoptext
